\starttypescript [math] [xits] [name]
  \definefontsynonym[MathRoman][name:xitsmath] [features=math\mathsizesuffix]
\stoptypescript

\starttypescript [serif] [xits] [name]
  \definefontsynonym[Serif]          [name:xitsregular]    [features=default]
  \definefontsynonym[SerifBold]      [name:xitsbold]       [features=default]
  \definefontsynonym[SerifItalic]    [name:xitsitalic]     [features=default]
  \definefontsynonym[SerifBoldItalic][name:xitsbolditalic] [features=default]
\stoptypescript

\starttypescript[xits]
   \definetypeface [xits] [rm] [serif] [xits] [default]
   \definetypeface [xits] [mm] [math]  [xits] [default]
\stoptypescript

\usetypescript[xits]
\setupbodyfont[xits]

\starttext

\section{Instalation}
TODO

\section{Usage}

\subsection{\LATEX}

\subsubsection{Requirements}

A modern \TEX\ engine with Unicode and OpenType suport is needed, namely
\LUATEX\ or \XETEX. In addition to {\tt fontspec} package, {\tt unicode-math}
is needed for using XITS fonts in math mode.

\subsubsection{Examples}
\startTEX
\documentclass{article}
\usepackage{unicode-math}
\setmainfont{XITS}
\setmathfont{XITS Math}

\begin{document}
Text $x+y=\sqrt{z}$
\end{document}
\stopTEX

\subsection{\CONTEXT}

\subsubsection{Requirements}
For using XITS in text mode, either \CONTEXT\ MkII with \XETEX\ engine, or
\CONTEXT\ MkIV are needed, but only MkIV supports using them in math mode.

\subsubsection{Examples}

\startTEX
\usetypescript[xits]
\setupbodyfont[xits]

\starttext
Text $x+y=\sqrt{z}$
\stoptext
\stopTEX

\section{Font features}
\definefontfeature[frac][default][frac=yes]
\definefontfeature[onum][default][onum=yes]
\starttable[|lT|l|r|]
\HL
\NC feature\NC Discription	\NC Example	 			\NC\SR
\HL
\NC onum \NC Oldstyle numbers	\NC {\addff{onum} 0123456789}		\NC\FR
\NC frac \NC Diagonal fractions	\NC {\addff{frac} 1/2 2/3 3/4 5/6 7/8}	\NC\LR
\HL
\stoptable

\stoptext
